%% Generated by Sphinx.
\def\sphinxdocclass{report}
\documentclass[letterpaper,10pt,english]{sphinxmanual}
\ifdefined\pdfpxdimen
   \let\sphinxpxdimen\pdfpxdimen\else\newdimen\sphinxpxdimen
\fi \sphinxpxdimen=.75bp\relax

\usepackage[utf8]{inputenc}
\ifdefined\DeclareUnicodeCharacter
 \ifdefined\DeclareUnicodeCharacterAsOptional
  \DeclareUnicodeCharacter{"00A0}{\nobreakspace}
  \DeclareUnicodeCharacter{"2500}{\sphinxunichar{2500}}
  \DeclareUnicodeCharacter{"2502}{\sphinxunichar{2502}}
  \DeclareUnicodeCharacter{"2514}{\sphinxunichar{2514}}
  \DeclareUnicodeCharacter{"251C}{\sphinxunichar{251C}}
  \DeclareUnicodeCharacter{"2572}{\textbackslash}
 \else
  \DeclareUnicodeCharacter{00A0}{\nobreakspace}
  \DeclareUnicodeCharacter{2500}{\sphinxunichar{2500}}
  \DeclareUnicodeCharacter{2502}{\sphinxunichar{2502}}
  \DeclareUnicodeCharacter{2514}{\sphinxunichar{2514}}
  \DeclareUnicodeCharacter{251C}{\sphinxunichar{251C}}
  \DeclareUnicodeCharacter{2572}{\textbackslash}
 \fi
\fi
\usepackage{cmap}
\usepackage[T1]{fontenc}
\usepackage{amsmath,amssymb,amstext}
\usepackage{babel}
\usepackage{times}
\usepackage[Bjarne]{fncychap}
\usepackage[dontkeepoldnames]{sphinx}

\usepackage{geometry}

% Include hyperref last.
\usepackage{hyperref}
% Fix anchor placement for figures with captions.
\usepackage{hypcap}% it must be loaded after hyperref.
% Set up styles of URL: it should be placed after hyperref.
\urlstyle{same}
\addto\captionsenglish{\renewcommand{\contentsname}{Contents:}}

\addto\captionsenglish{\renewcommand{\figurename}{Fig.}}
\addto\captionsenglish{\renewcommand{\tablename}{Table}}
\addto\captionsenglish{\renewcommand{\literalblockname}{Listing}}

\addto\captionsenglish{\renewcommand{\literalblockcontinuedname}{continued from previous page}}
\addto\captionsenglish{\renewcommand{\literalblockcontinuesname}{continues on next page}}

\addto\extrasenglish{\def\pageautorefname{page}}

\setcounter{tocdepth}{1}



\title{brainSimulator Documentation}
\date{Nov 13, 2017}
\release{0.5.1}
\author{SiPBA@UGR}
\newcommand{\sphinxlogo}{\vbox{}}
\renewcommand{\releasename}{Release}
\makeindex

\begin{document}

\maketitle
\sphinxtableofcontents
\phantomsection\label{\detokenize{index::doc}}


Functional brain image synthesis using the KDE or MVN distribution. Currently in beta. Python code.

\sphinxtitleref{brainSimulator} is a brain image synthesis procedure intended to generate a new image set that share characteristics with an original one. The system focuses on nuclear imaging modalities such as PET or SPECT brain images. It analyses the dataset by applying PCA to the original dataset, and then model the distribution of samples in the projected eigenbrain space using a Probability Density Function (PDF) estimator. Once the model has been built, anyone can generate new coordinates on the eigenbrain space belonging to the same class, which can be then projected back to the image space.
\phantomsection\label{\detokenize{index:module-brainSimulator}}\index{brainSimulator (module)}\phantomsection\label{\detokenize{index:module-brainSimulator}}\index{brainSimulator (module)}
Created on Thu Apr 28 15:53:15 2016
Last update: 9 Aug, 2017

@author: Francisco J. Martinez-Murcia \textless{}\sphinxhref{mailto:fjesusmartinez@ugr.es}{fjesusmartinez@ugr.es}\textgreater{}

Copyright (C) 2017 Francisco Jesús Martínez Murcia and SiPBA Research Group

This program is free software: you can redistribute it and/or modify
it under the terms of the GNU General Public License as published by
the Free Software Foundation, either version 3 of the License, or
(at your option) any later version.

This program is distributed in the hope that it will be useful,
but WITHOUT ANY WARRANTY; without even the implied warranty of
MERCHANTABILITY or FITNESS FOR A PARTICULAR PURPOSE.  See the
GNU General Public License for more details.

You should have received a copy of the GNU General Public License
along with this program.  If not, see \textless{}\sphinxurl{http://www.gnu.org/licenses/}\textgreater{}.
\index{KDEestimator (class in brainSimulator)}

\begin{fulllineitems}
\phantomsection\label{\detokenize{index:brainSimulator.KDEestimator}}\pysiglinewithargsret{\sphinxbfcode{class }\sphinxcode{brainSimulator.}\sphinxbfcode{KDEestimator}}{\emph{bandwidth=1.0}}{}
An interface for generating random numbers according
to a given Kernel Density Estimation (KDE) parametrization based on the 
data.
\index{botev\_bandwidth() (brainSimulator.KDEestimator method)}

\begin{fulllineitems}
\phantomsection\label{\detokenize{index:brainSimulator.KDEestimator.botev_bandwidth}}\pysiglinewithargsret{\sphinxbfcode{botev\_bandwidth}}{\emph{data}}{}
Implementation of the KDE bandwidth selection method outline in:
\begin{enumerate}
\setcounter{enumi}{25}
\item {} \begin{enumerate}
\item {} 
Botev, J. F. Grotowski, and D. P. Kroese. \sphinxstyleemphasis{Kernel density estimation via diffusion.} The Annals of Statistics, 38(5):2916-2957, 2010.

\end{enumerate}

\end{enumerate}

Based on the implementation of Daniel B. Smith, PhD. The object is a callable returning the bandwidth for a 1D kernel.

Forked from the package \sphinxtitleref{PyQT\_fit \textless{}https://code.google.com/archive/p/pyqt-fit/\textgreater{}}.
\begin{quote}\begin{description}
\item[{Parameters}] \leavevmode
\sphinxstyleliteralstrong{data} (\sphinxstyleliteralemphasis{numpy.ndarray}) \textendash{} 1D array containing the data to model with a 1D KDE.

\item[{Returns}] \leavevmode
Optimal bandwidth according to the data.

\end{description}\end{quote}

\end{fulllineitems}

\index{finite() (brainSimulator.KDEestimator method)}

\begin{fulllineitems}
\phantomsection\label{\detokenize{index:brainSimulator.KDEestimator.finite}}\pysiglinewithargsret{\sphinxbfcode{finite}}{\emph{val}}{}
Checks if a value is finite or not

\end{fulllineitems}


\end{fulllineitems}

\index{BrainSimulator (class in brainSimulator)}

\begin{fulllineitems}
\phantomsection\label{\detokenize{index:brainSimulator.BrainSimulator}}\pysiglinewithargsret{\sphinxbfcode{class }\sphinxcode{brainSimulator.}\sphinxbfcode{BrainSimulator}}{\emph{method='kde'}, \emph{algorithm='PCA'}, \emph{N=100}, \emph{n\_comp=-1}, \emph{regularize=False}, \emph{verbose=False}}{}~\index{createNewBrains() (brainSimulator.BrainSimulator method)}

\begin{fulllineitems}
\phantomsection\label{\detokenize{index:brainSimulator.BrainSimulator.createNewBrains}}\pysiglinewithargsret{\sphinxbfcode{createNewBrains}}{\emph{N}, \emph{kernel}, \emph{components=None}}{}
Creates new samples from the model.

\end{fulllineitems}

\index{generateDataset() (brainSimulator.BrainSimulator method)}

\begin{fulllineitems}
\phantomsection\label{\detokenize{index:brainSimulator.BrainSimulator.generateDataset}}\pysiglinewithargsret{\sphinxbfcode{generateDataset}}{\emph{stack}, \emph{labels}, \emph{N=100}, \emph{classes=None}, \emph{components=None}}{}
Fits the model and generates a new set of N elements for each class
specified in “classes”.
\begin{quote}\begin{description}
\item[{Parameters}] \leavevmode\begin{itemize}
\item {} 
\sphinxstyleliteralstrong{stack} (\sphinxstyleliteralemphasis{numpy.ndarray}) \textendash{} the stack from which the model will be created

\item {} 
\sphinxstyleliteralstrong{labels} (\sphinxstyleliteralemphasis{numpy.ndarray}) \textendash{} a vector containing the labels of the stacked dataset

\item {} 
\sphinxstyleliteralstrong{N} (either int (the same N will be generated per class) or a list of the same length as \sphinxtitleref{classes} containing the number of subjects to be generated for each class respectively.) \textendash{} the number of elements (per class) to be generated

\item {} 
\sphinxstyleliteralstrong{classes} (a list of the classes to be generated, e.g.: \sphinxtitleref{{[}0, 2{]}} or \sphinxtitleref{{[}‘AD’, ‘CTL’{]}}.) \textendash{} the classes that we aim to generate

\item {} 
\sphinxstyleliteralstrong{components} (\sphinxstyleliteralemphasis{integer}) \textendash{} the number of components used in the synthesis. This parameter is only valid if \sphinxtitleref{components} here is smaller than the \sphinxtitleref{n\_comp} specified when creating and fitting the {\color{red}\bfseries{}{}`}BrainSimulator{}`object.

\end{itemize}

\item[{Returns}] \leavevmode
array with labels of the synthetic stack

\end{description}\end{quote}

\end{fulllineitems}


\end{fulllineitems}



\chapter{Indices and tables}
\label{\detokenize{index:welcome-to-brainsimulator-s-documentation}}\label{\detokenize{index:indices-and-tables}}\begin{itemize}
\item {} 
\DUrole{xref,std,std-ref}{genindex}

\item {} 
\DUrole{xref,std,std-ref}{modindex}

\item {} 
\DUrole{xref,std,std-ref}{search}

\end{itemize}


\renewcommand{\indexname}{Python Module Index}
\begin{sphinxtheindex}
\def\bigletter#1{{\Large\sffamily#1}\nopagebreak\vspace{1mm}}
\bigletter{b}
\item {\sphinxstyleindexentry{brainSimulator}}\sphinxstyleindexextra{Unix, Windows}\sphinxstyleindexpageref{index:\detokenize{module-brainSimulator}}
\end{sphinxtheindex}

\renewcommand{\indexname}{Index}
\printindex
\end{document}